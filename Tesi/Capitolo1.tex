\chapter{Introduzione}\label{chapintroduzione}

\section{Motivazione e contesto applicativo}\label{secmotivazione-contesto}
SOULFRAME nasce da un problema concreto: in molti ambienti VR la scena è credibile, ma la conversazione con l'agente resta fragile o troppo rigida. Questa tesi propone un prototipo in cui l'utente parla con un avatar embodied e riceve una risposta vocale contestuale nella stessa scena 3D.

La difficoltà è tenere insieme tre piani che spesso vengono trattati separatamente: interazione vocale a turni, continuità del dialogo e presenza dell'interlocutore virtuale. In VR non basta la qualità grafica: la percezione di presenza dipende anche dalla coerenza tra azioni, tempi di risposta ed eventi della scena \cite{Slater2016VR}. Quando questi elementi si disallineano, l'agente appare poco affidabile anche se il contenuto linguistico è corretto.

Le evidenze sperimentali sugli ECA vocali indicano che STT+TTS in tempo reale può aumentare la co-presenza rispetto a soluzioni preregistrate \cite{Kan2023ECA_Training}. La letteratura in XR mostra inoltre che la maggior parte dei prototipi è sviluppata in VR, spesso con Unity, e che cresce l'interesse per dialoghi più adattivi supportati da modelli neurali \cite{Yang2025ECA_XR}. In SOULFRAME questa linea viene applicata a una configurazione senza HMD obbligatorio, orientata alla fruizione via browser.

L'idea chiave del progetto è trattare l'avatar come un profilo persistente composto da aspetto, voce e memoria. Così l'interazione non riparte da zero a ogni turno e mantiene una continuità minima tra sessioni. Nel Capitolo~\ref{chap:stato-arte}, Figura~\ref{fig:ca-evoluzione}, è richiamata la traiettoria dei Conversational Agents dalla fase scriptata alle soluzioni recenti basate su LLM.

In questo contesto SOULFRAME non è una demo grafica isolata: combina pipeline vocale, memoria conversazionale e resa embodied per verificare quanto questi elementi incidano sulla qualità percepita dell'interazione.

\section{Perimetro e requisiti di progetto}\label{secperimetro-requisiti}
SOULFRAME è un prototipo di interazione immersiva con ECA vocale embodied in ambiente 3D. L'obiettivo è valutare fattibilità tecnica e qualità percepita in scenari con componente sociale. Non è una piattaforma generale per creare mondi virtuali e non è una soluzione clinica certificata.

Il perimetro resta concentrato sull'integrazione \textit{end-to-end} tra voce, memoria conversazionale e rappresentazione embodied dell'agente. Ho scelto di vincolare l'accesso alla conversazione operativa a un onboarding minimo del profilo avatar, perché senza voce e memoria l'interazione diventa rapidamente \textit{stateless}.

Sul piano funzionale, il nucleo è la pipeline a turni \textit{push-to-talk}: acquisizione audio, trascrizione STT, generazione contestuale con RAG/LLM e risposta TTS. Funzioni come descrizione immagini e OCR rientrano nel prototipo come capacità di supporto alla memoria; i dettagli implementativi sono discussi nei capitoli successivi.

I requisiti non funzionali riguardano soprattutto latenza percepita, robustezza ai guasti parziali e modularità dei servizi. La tesi adotta quindi criteri operativi verificabili sul prototipo, evitando soglie astratte non allineate al contesto d'uso reale.

\section{Obiettivi e contributi di SOULFRAME}\label{secobiettivi-contributi}
Gli obiettivi di SOULFRAME si collocano nel filone degli ECA in XR, dove sono comuni implementazioni VR in Unity e interazioni vocali, ma con metodi di valutazione ancora eterogenei \cite{Yang2025ECA_XR}. Il progetto punta a costruire un prototipo in cui l'integrazione \textit{end-to-end} sia osservabile e discutibile con criteri espliciti.

RQ1 riguarda la fattibilità tecnica di un'interazione vocale in tempo reale con un agente embodied in ambiente immersivo 3D. In termini operativi, la catena STT $\rightarrow$ RAG/LLM $\rightarrow$ TTS $\rightarrow$ output audiovisivo deve funzionare in modo continuo, includendo anche le capacità di supporto (descrizione immagini e OCR documentale). La valutazione considera tempi per blocco, completamento dei turni senza errori bloccanti e coerenza delle transizioni tra ascolto, elaborazione e risposta.

RQ2 riguarda la qualità percepita dell'interazione, con attenzione a co-presenza e naturalezza dialogica. SOULFRAME prevede una valutazione qualitativa guidata, affiancata quando possibile da metriche soggettive usate negli studi VR con ECA. Studi comparativi tra agenti conversazionali e audio pre-scriptato mostrano differenze osservabili, soprattutto sulla co-presenza \cite{Kan2023ECA_Training}.

RQ3 riguarda modularità e riproducibilità della pipeline. Il sistema deve permettere la sostituzione dei componenti principali senza riscrivere l'intera architettura e rendere tracciabili configurazioni, tempi e risultati delle prove. Figura~\ref{fig:obiettivi-contributi} sintetizza la corrispondenza tra i tre obiettivi e i contributi attesi.

\begin{figure}[t]
\centering
\includegraphics[width=0.85\textwidth]{obiettivi_contributi.png}
\caption{Corrispondenza tra obiettivi di ricerca e contributi del progetto SOULFRAME.}
\label{fig:obiettivi-contributi}
\end{figure}

Sul piano tecnico, il progetto introduce un'architettura modulare \textit{end-to-end} basata su micro-servizi (STT, RAG/LLM, TTS e memoria) con interfacce API esplicite. Sul piano sperimentale, il risultato è un \textit{proof-of-concept} verificabile con un set minimo di misure soggettive e osservazioni qualitative. Per la replicabilità, resta centrale la documentazione delle scelte progettuali, in linea con il dibattito sul \textit{dialogue management} e sulla necessità di metriche più comparabili \cite{Laranjo2018CA}.

\section{Panoramica del sistema e flusso end-to-end}\label{secpanoramica-sistema}
SOULFRAME adotta un'architettura client--server per sostenere un dialogo vocale a turni in ambiente immersivo 3D. Il client Unity WebGL gestisce scena, avatar e interfaccia; il backend gestisce trascrizione, memoria conversazionale, generazione testuale e sintesi vocale.

Il flusso è lineare a livello concettuale: input vocale dell'utente, trascrizione STT, generazione contestuale con RAG/LLM, risposta TTS e playback sull'avatar. Le stesse fasi devono però restare coordinate nei tempi, perché ritardi o transizioni incoerenti compromettono la naturalezza percepita.

Ho scelto una fruizione via browser WebGL senza HMD obbligatorio per abbassare la barriera di accesso durante sviluppo e test. In parallelo ho mantenuto una pipeline modulare a micro-servizi, così da poter sostituire i blocchi principali senza riscrivere il client.

La vista d'insieme architetturale è riportata nel Capitolo~\ref{chap:architettura}, in Figura~\ref{fig:pipeline-e2e}, dove la pipeline è discussa con maggior dettaglio tecnico.

\section{Metodo di lavoro e struttura della tesi}\label{secmetodo-struttura}

\begin{figure}[t]
\centering
\includegraphics[width=0.85\textwidth]{struttura_tesi.png}
\caption{Struttura della tesi e organizzazione dei capitoli.}
\label{fig:struttura-tesi}
\end{figure}

Lo sviluppo di SOULFRAME segue un approccio iterativo basato su prototipazione incrementale. Questa scelta riduce il rischio tecnico tipico dei sistemi che combinano reattività a turni (\textit{push-to-talk}), elaborazione linguistica e rendering immersivo. Il lavoro procede per integrazioni successive: prima validazione dei moduli in isolamento, poi integrazione della pipeline completa e verifica su scenari progressivamente più complessi. Durante il processo vengono tracciati compromessi e dipendenze per mantenere l'evoluzione del sistema leggibile e replicabile.

La tesi segue una progressione lineare: dai fondamenti teorici si passa all'architettura, poi alle scelte implementative e infine alla valutazione sperimentale del prototipo. L'ultimo passaggio raccoglie limiti emersi e sviluppi futuri, mantenendo continuità tra decisioni progettuali e risultati osservati. La sequenza è stata pensata per separare chiaramente il ``perché'' delle scelte dal ``come'' tecnico con cui sono state realizzate.

Figura~\ref{fig:struttura-tesi} offre una vista sintetica della struttura e della progressione logica tra capitoli.

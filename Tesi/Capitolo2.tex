\chapter{Fondamenti e Stato dell'Arte}
\label{chap:stato-arte}

\section{Agenti conversazionali embodied in XR}
\label{sec:eca-xr}
\subsection{Presenza sociale, co-presenza e ruolo della voce}
Contenuto in preparazione.
\subsection{Limiti aperti nei sistemi conversazionali immersivi}
Contenuto in preparazione.
% TODO: riprendere 2-3 studi già citati nel Capitolo 1 e sintetizzare le metriche usate (presenza, co-presenza, naturalezza).

\section{Pipeline AI adottata in SOULFRAME}
\label{sec:pipeline-ai}
\subsection{Speech-to-Text con Whisper}
Contenuto in preparazione.
\subsection{Memoria conversazionale con RAG (LLM + embeddings + retrieval)}
Contenuto in preparazione.
\subsection{Text-to-Speech con Coqui XTTS v2}
Contenuto in preparazione.
% TODO: aggiungere una tabella breve (modulo, libreria/modello, motivazione della scelta, limite noto).

\section{Integrazione client-server del sistema}
\label{sec:integrazione-client-server}
\subsection{Frontend Unity WebGL e principali vincoli di piattaforma}
Contenuto in preparazione.
\subsection{Backend FastAPI a micro-servizi}
Contenuto in preparazione.
\subsection{Comunicazione HTTP, CORS e proxy applicativo}
Contenuto in preparazione.
% TODO: descrivere un flusso end-to-end reale (audio utente -> /transcribe -> /chat o /recall -> /tts_stream) e indicare il comportamento in caso di servizio non disponibile.

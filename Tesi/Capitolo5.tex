\chapter{Risultati e Valutazione}
\label{chap:risultati}

\section{Impostazione della valutazione}
\label{sec:metodologia}
\subsection{Scenari di prova e setup sperimentale}
Contenuto in preparazione.
\subsection{Metriche tecniche adottate}
Contenuto in preparazione.
\subsection{Metriche di esperienza utente}
Contenuto in preparazione.
% TODO facoltativo: se si attiva lo studio utenti, definire protocollo, numero partecipanti, età media, durata sessione e criteri di inclusione/esclusione.

\section{Risultati tecnici del prototipo}
\label{sec:quantitativi}
\subsection{Prestazioni della pipeline STT-RAG-TTS}
Contenuto in preparazione.
\subsection{Latenza end-to-end e stabilità dei servizi}
Contenuto in preparazione.
\subsection{Osservazioni tra ambiente locale e server}
Contenuto in preparazione.
% TODO: inserire una tabella unica con min/medio/max per fase, per confrontare in modo diretto ambiente locale e server.

\section{Risultati qualitativi e casi d'uso}
\label{sec:qualitativi}
\subsection{Qualità percepita dell'interazione}
Contenuto in preparazione.
\subsection{Usabilità interfaccia desktop e touch}
Contenuto in preparazione.
\subsection{Analisi di casi e failure cases}
Contenuto in preparazione.
% TODO: selezionare 2-3 casi rappresentativi (positivo, intermedio, critico) e motivarne la scelta.

\section{Valutazione utenti (estensione facoltativa)}
\label{sec:utenti-facoltativa}
\subsection{Risultati SUS}
Contenuto in preparazione.
\subsection{Risultati NPS}
Contenuto in preparazione.
\subsection{Confronti tra gruppi}
Contenuto in preparazione.
% TODO facoltativo: mantenere questa sezione solo con campione sufficiente e bilanciato; in caso contrario, esplicitare il limite metodologico.

\section{Discussione dei risultati}
\label{sec:discussione}
\subsection{Punti di forza}
Contenuto in preparazione.
\subsection{Limiti emersi}
Contenuto in preparazione.
\subsection{Sintesi rispetto alle research questions}
Contenuto in preparazione.

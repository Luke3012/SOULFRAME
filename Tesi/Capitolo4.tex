\chapter{Sviluppo del Progetto: Implementazione e Criticità}
\label{chap:implementazione}

\section{Implementazione frontend}
\label{sec:impl-frontend}
\subsection{Gestione stati UI e navigazione}
Contenuto in preparazione.
\subsection{Gestione avatar e onboarding}
Contenuto in preparazione.
\subsection{Acquisizione audio e input desktop/touch}
Contenuto in preparazione.
\subsection{MainMode conversazionale}
Contenuto in preparazione.
% TODO: citare solo i metodi principali di UIFlowController, AvatarManager e AudioRecorder, indicando inizio e fine delle fasi del flusso utente.

\section{Implementazione backend}
\label{sec:impl-backend}
\subsection{Servizio STT}
Contenuto in preparazione.
\subsection{Servizio RAG e memoria per avatar}
Contenuto in preparazione.
\subsection{Servizio TTS e streaming audio}
Contenuto in preparazione.
\subsection{Servizio asset avatar}
Contenuto in preparazione.
% TODO: inserire una tabella con gli endpoint effettivamente chiamati dal client (/transcribe, /chat, /tts_stream, /avatars/list, /ingest_file), con input/output minimo atteso.

\section{Integrazione end-to-end}
\label{sec:integrazione}
\subsection{Orchestrazione richieste tra client, proxy e micro-servizi}
Contenuto in preparazione.
\subsection{Normalizzazione endpoint locale vs produzione}
Contenuto in preparazione.
\subsection{Gestione errori, retry e fallback}
Contenuto in preparazione.

\section{Criticità affrontate e soluzioni}
\label{sec:problemi-soluzioni}
\subsection{Latenza e timeout}
Contenuto in preparazione.
\subsection{CORS e routing API}
Contenuto in preparazione.
\subsection{OCR e qualità dell'ingestione}
Contenuto in preparazione.
\subsection{Compatibilità dipendenze/modelli}
Contenuto in preparazione.
\subsection{Differenze operative tra Windows e Ubuntu}
Contenuto in preparazione.
% TODO: per ogni criticità usare lo schema "problema -> causa -> soluzione -> impatto".

\section{Runbook operativo essenziale}
\label{sec:runbook}
Contenuto in preparazione.
% TODO: inserire un mini runbook d'emergenza (servizio non disponibile, errore CORS, timeout TTS) con i comandi minimi in ordine di esecuzione.

\section{Affidabilità e sicurezza operativa}
\label{sec:sicurezza}
Contenuto in preparazione.
% TODO: distinguere chiaramente ciò che è già implementato (validazioni base, log, isolamento per avatar) da ciò che resta da hardenizzare.

\chapter{Architettura e Tecnologie Utilizzate}
\label{chap:architettura}

\section{Requisiti del sistema}
\label{sec:requisiti}
\subsection{Requisiti funzionali}
Contenuto in preparazione.
\subsection{Requisiti non funzionali}
Contenuto in preparazione.
% TODO: riportare 5 requisiti funzionali concreti (es. creazione avatar, setup voce, ingest memoria, conversazione push-to-talk, playback TTS) e 3 requisiti non funzionali misurabili.

\section{Architettura di riferimento di SOULFRAME}
\label{sec:architettura-riferimento}
\subsection{Vista d'insieme dei componenti frontend/backend}
Contenuto in preparazione.
\subsection{\texorpdfstring{Flusso end-to-end audio $\rightarrow$ testo $\rightarrow$ risposta $\rightarrow$ audio}{Flusso end-to-end audio -> testo -> risposta -> audio}}
Contenuto in preparazione.
\subsection{Separazione tra modalità locale e modalità produzione}
Contenuto in preparazione.
% TODO: aggiungere un paragrafo sintetico sulle differenze operative tra locale e produzione (URL endpoint, Caddy, systemd, percorsi dati).

\section{Componenti implementati}
\label{sec:componenti-implementati}
\subsection{Frontend Unity WebGL}
Contenuto in preparazione.
\subsection{Backend AI a micro-servizi (Whisper, RAG, TTS, Avatar Asset)}
Contenuto in preparazione.
\subsection{Persistenza e gestione dati per avatar}
Contenuto in preparazione.
% TODO: inserire una tabella "componente -> file principali -> endpoint/porte -> responsabilità".

\section{Setup e deploy operativo}
\label{sec:setup-deploy}
\subsection{Ambiente locale Windows}
Contenuto in preparazione.
\subsection{Ambiente server Ubuntu}
Contenuto in preparazione.
\subsection{Servizi di supporto (systemd, Caddy, script amministrativi)}
Contenuto in preparazione.
% TODO: documentare la procedura realmente eseguita (start, stop, update, rollback), con criticità incontrate e verifiche effettuate.

\chapter*{Ringraziamenti}
\addcontentsline{toc}{chapter}{Ringraziamenti}
\label{chap:ringraziamenti}

Scrivere questi ringraziamenti è strano: sulla carta è la parte più "leggera" della tesi, ma è anche quella in cui non posso nascondermi dietro la tecnica. Qui non devo dimostrare niente, devo solo essere onesto.

Ringrazio prima di tutto me stesso. Sono partito alla grande, poi mi sono perso per strada: per necessità, per lavoro, per la vita che a volte decide i tempi al posto tuo. Ci ho messo cinque anni e passa e, sì, ho pensato di mollare tutto. Non per pigrizia, ma perché a un certo punto mi sembrava di stare rincorrendo una versione di me rimasta indietro. Per molto tempo mi sono sentito in colpa per l'essere fuori corso, come se avessi "sbagliato" qualcosa. Oggi la leggo diversamente: ho fatto quello che potevo, con quello che avevo, e soprattutto ho trovato un modo per riprendermi. Questa tesi per me è un punto di partenza: non per diventare un altro, ma per continuare senza sentirmi bloccato.

Ringrazio la mia relatrice, la Prof.ssa Paola Barra. Anche se siamo riusciti a vederci spesso di sfuggita, per me è stato davvero bello incontrarla: è una di quelle persone che si percepiscono "belle" nel modo più semplice e più raro del termine. Le auguro, come le ho sempre detto, di stare sempre bene e tranquilla. Ringrazio anche Ilaria e Attilio: due presenze inaspettate, preziose, che mi hanno incitato e incoraggiato nei momenti giusti. Li stimo molto, e mi porto dietro la loro buona energia.

Ringrazio i miei nonni, perché in modi diversi sono stati casa, presenza e sostegno. Ringrazio mio padre, per esserci stato e per tutto quello che mi ha insegnato, anche quando io non ero bravo a riconoscerlo. Ringrazio anche tutto ciò che mi ha formato, nel bene e nel male: alcune cose mi hanno dato slancio, altre mi hanno costretto a crescere. Non è stato sempre comodo, ma è stato reale.

Ringrazio le persone che ho incontrato lungo il percorso, perché spesso sono state loro a darmi coraggio quando io ne avevo poco. Un grazie enorme a Marco, per avermi sopportato, spronato e rimesso in riga con pazienza (anche quando ero intrattabile). Grazie ad Antonio per quei "complimenti" ogni tanto, chiaramente al contrario, ma comunque capaci di farmi sorridere quando serviva. Ringrazio anche Vincenzo per essere un bruscianese d'eccellenza, per essere uno dei volti biondi del progetto e per avermi accolto con una naturalezza che non dimentico: grazie davvero, e sappi che io farò lo stesso, sempre. Ringrazio Daniele e Federica per le pizze buone e, soprattutto, per avermi insegnato una tecnica sottile ma importantissima: essere un pelino più leggero, quando serve. Ringrazio Roberto e Roberta, i due roberti, per essere affascinanti e per quella presenza che resta, anche quando non si fa rumore.

Grazie ai colleghi, in particolare Vincenzo e Simone: sostenersi a vicenda non è mai abbastanza, eppure per un po' lo è stato. I chiarimenti anche a orari assurdi, le risate quando eravamo cotti, e quei pasti post-esame esageratamente calorici che per un periodo sono stati una specie di rito.

Ringrazio anche Livio e Francesca, il mio vecchio datore di lavoro e sua moglie: per quello che mi hanno dato, per quello che mi hanno insegnato e per l'aiuto nei momenti bui. E ringrazio tutte le persone che ci sono state, in qualsiasi forma: con un messaggio, una battuta, una spinta, una mano tesa. Anche quando magari non lo sapevano, hanno contato.

Un ringraziamento, un po' storto ma sincero, va anche alla mia ansia. Mi ha bloccato tante volte, però in questi anni mi sono fatto forza davvero. Oggi capita sempre più spesso che sia lei ad avere paura di me. Non so come andrà il giorno della presentazione (magari la voce trema, magari no), ma so che non sono più la stessa persona che ha iniziato. E questo, da solo, vale tanto.

Infine ringrazio \textbf{SOULFRAME}. Non è stato solo un progetto: per me è rimasto un simbolo. È stata una sfida tecnica e personale, anche per la complessità e per tutto quello che mi ha costretto a reimparare da capo. L'ho vissuta con quella tipica dose di ossessione e attaccatura che mi prende quando ci tengo davvero, ma proprio per questo è una delle cose di cui vado più fiero.

Ci ho messo tempo, sì. Ma oggi posso dirlo: ne è valsa la pena.
